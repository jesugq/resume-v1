%% Document Packages
\documentclass{res}
\usepackage[utf8]{inputenc}
\usepackage[english]{babel}
\usepackage{graphicx}
\usepackage{hyperref}
\usepackage{tabularx}
\usepackage{xcolor}
\usepackage{mwe}
%% Document Commands
% Raise an image's position to match the center of text.
% @param scale      The scale for this image.
% @param filename   The filename for this image.
\newcommand{\raiseimage}[2]{
    \raisebox{-0.35\height}{\includegraphics[scale=#1]{#2}}
}
% Present social information using an image.
% @param value      The value for this social gateway.
\newcommand{\github}[1]{
    \raiseimage{0.050}{./images/mark-github.png}
    \href{http://www.github.com/#1}{github.com/#1}
}
\newcommand{\mail}[1]{
    \raiseimage{0.050}{./images/mail.png}
    \href{mailto://#1}{#1}
}
\newcommand{\mobile}[1]{
    \raiseimage{0.030}{./images/device-mobile.png}
    #1
}
\newcommand{\separator}[0]{
    \raisebox{3\height}{\rule{13.75cm}{1mm}}
    \\
    \hfill
}

%% Document Properties
\hypersetup{colorlinks=true, urlcolor=gray}
\renewcommand\labelitemi{\guillemotright}
\renewcommand{\arraystretch}{0}
\newcolumntype{B}{X}
\newcolumntype{S}{>{\hsize=.3\hsize}X}
\begin{document}

%% Document Header
\name{
    JESÚS ANTONIO GONZÁLEZ QUEVEDO
    \jump{2}
}
\address{
    \mobile{(+52) 271 108 5129}
    \mail{jesugq@gmail.com}
    \github{jesugq}
}

%% Document Resume
\begin{resume}
    \separator

    \section{\large{EDUCACIÓN}} \jump{2}
    \textbf{Ingeniero en Tecnologías Computacionales} \jump{1}
    \yearrange{2015/08 - Actual} \\
    Instituto Tecnológico y de Estudios Superiores de Monterrey.
    
    \longjump

    \section{\large{PROYECTOS PERSONALES}} \jump{2}
    \textbf{Kubeet-Landing | AngularJS} \jump{1}
    \yearrange{2019/02 - 2019/05}
    \begin{itemize}
        \item Desarrollo de un catálogo en línea para la venta de material de autoenseñanza.
        \item Funcionalidad de carrito de compras utilizando un sistema de CRUD.
        \item Implementación y administración de la base de datos no relacional Firebase.
        \item \github{jesugq/kubeet-landing}
        \begin{itemize}
            \item Angular 7 | Firebase | HTML | Typescript |
        \end{itemize}
    \end{itemize}

    \textbf{WCrafter iOS | Swift} \jump{1}
    \yearrange{2018/02 - 2018/05}
    \begin{itemize}
        \item Desarollo de una aplicación de monitoreo de autobúses para la plataforma iOS.
        \item Uso de los componentes visuales integrados en el IDE oficial de Apple.
        \item \github{jesugq/WCrafter-iOS}
        \begin{itemize}
            \item Swift | XCode |
        \end{itemize}
    \end{itemize}

    \textbf{Tutor en Línea | Microsoft Office} \jump{1}
    \yearrange{2016/01 - 2016/04}
    \begin{itemize}
        \item Seguimiento de estudiantes durante su aprendizaje de Microsoft Word y Excel.
        \item Uso de la plataforma PrepaNet del Tecnológico de Monterrey.
    \end{itemize}

    \longjump

    \section{\large{COMPETENCIAS}} \jump{3}
    \begin{tabularx}{15cm}{SB}
        \rightline{\textbf{Programación}} &
        Java, C, Javascript, Typescript, LaTeX
        \\
        \rightline{\textbf{Frontend y Backend}} &
        Angular 7, React, Firebase, CSS
        \\
        \rightline{\textbf{Bases de Datos}} &
        MySQL, Cisco IOS
        \\
        \rightline{\textbf{Línea de Comando}} &
        Git, Bash / Shell, Npm, Yarn, Linux
        \\
        \rightline{\textbf{IDEs y Desarollo}} &
        VSCode, NetBeans, Eclipse, Android Studio, XCode
    \end{tabularx}
\end{resume}

\end{document}