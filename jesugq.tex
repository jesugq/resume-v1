%% Document Packages
\documentclass{res}
\usepackage[utf8]{inputenc}
\usepackage[english]{babel}
\usepackage{graphicx}
\usepackage{hyperref}
\usepackage{tabularx}
\usepackage{xcolor}
\usepackage{mwe}
%% Document Commands
% Raise an image's position to match the center of text.
% @param scale      The scale for this image.
% @param filename   The filename for this image.
\newcommand{\raiseimage}[2]{
    \raisebox{-0.2\height}{\includegraphics[scale=#1]{#2}}
}

% Present social information using an image.
% @param value      The value for this social gateway.
\newcommand{\github}[1]{
    \raiseimage{0.030}{./images/mark-github.png}
    \href{http://www.github.com/#1}{github.com/#1}
}
\newcommand{\mail}[1]{
    \raiseimage{0.030}{./images/mail.png}
    \href{mailto://#1}{#1}
}
\newcommand{\mobile}[1]{
    \raiseimage{0.015}{./images/device-mobile.png}
    #1
}

% Separator line between the header and the resume.
\newcommand{\separator}[0]{
    \rule{13.75cm}{0.75mm}
}

% Year inclusion with a line break.
% @param range      The year in which the activity took part in.
\newcommand{\yearrange}[1]{
    \hfill
    #1
    \hspace{13mm}
}

%% Add a little more vertical space.
\newcommand{\jump}[1]{
    \vspace{#1mm}
}
\newcommand{\longjump}[0]{
    \vspace{5mm}
}

%% Document Properties
\hypersetup{colorlinks=true, urlcolor=gray}
\renewcommand\labelitemi{\guillemotright}
\renewcommand{\arraystretch}{0}
\newcolumntype{B}{X}
\newcolumntype{S}{>{\hsize=.3\hsize}X}
\begin{document}

%% Document Header
\name{
    JESÚS ANTONIO GONZÁLEZ QUEVEDO
    \jump{2}
}
\address{
    \mobile{(+52) 271 108 5129}
    \mail{jesugq@gmail.com}
    \github{jesugq}
}

%% Document Resume
\begin{resume}
    \vspace{-4mm}
    \separator

    \section{\large{EDUCACIÓN}} \jump{2}

    \textbf{Ingeniero en Tecnologías Computacionales} \jump{1}
    \yearrange{2015/08 - Actual} \\
    Instituto Tecnológico y de Estudios Superiores de Monterrey.
    
    \section{\large{EXPERIENCIA LABORAL}} \jump{2}

    \textbf{Vehitfer : Inment Development} \jump{1}
    \yearrange{2019/09 - 2019/10}
    \begin{itemize}
        \item Desarrollo de un componente para la presentación de una ficha técnica en Angular 8.
        \item Vínculo con la base de datos no relacional Graphql en la mutación de actualización.
        \item Implementación del servicio utilizando el sistema de consulta de datos Apollo.
    \end{itemize}

    \textbf{Enfermería : Inment Development} \jump{1}
    \yearrange{2019/10 - Actual}

    \section{\large{PROYECTOS PERSONALES}} \jump{2}

    \textbf{Kubeet-Landing} \jump{1}
    \yearrange{2019/02 - 2019/05}
    \begin{itemize}
        \item Creación de un catálogo en línea para la venta de material de autoenseñanza.
        \item Funcionalidad de un carrito de compras utilizando el sistema CRUD.
        \item Administración de la base de datos no relacional Google Firebase.
    \end{itemize}

    \textbf{WCrafter iOS} \jump{1}
    \yearrange{2018/02 - 2018/05}
    \begin{itemize}
        \item Desarollo de una aplicación de monitoreo de autobúses para la plataforma iOS.
        \item Uso de los componentes visuales integrados en XCode, Swift.
    \end{itemize}

    \textbf{Tutor en Línea de Microsoft Office} \jump{1}
    \yearrange{2016/01 - 2016/04}
    \begin{itemize}
        \item Seguimiento de estudiantes durante su aprendizaje de Microsoft Word y Excel.
        \item Uso de la plataforma PrepaNet del Tecnológico de Monterrey.
    \end{itemize}

    \section{\large{COMPETENCIAS}} \jump{3}

    \begin{tabularx}{15cm}{SB}
        \leftline{\textbf{Programación}} &
        Java, C, Javascript, Typescript, LaTeX
        \\
        \leftline{\textbf{Web Development}} &
        Angular, React, Firebase, HML \& CSS, Graphql
        \\
        \leftline{\textbf{Bases de Datos}} &
        MySQL, Cisco IOS
        \\
        \leftline{\textbf{Línea de Comando}} &
        Git, Bash / Shell, nvm, npm, yarn
        \\
        \leftline{\textbf{IDEs y Desarollo}} &
        VSCode, GitKraken, NetBeans, Eclipse, Android Studio, XCode
    \end{tabularx}
\end{resume}

\end{document}